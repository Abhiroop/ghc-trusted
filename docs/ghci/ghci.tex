%
% (c) The OBFUSCATION-THROUGH-GRATUITOUS-PREPROCESSOR-ABUSE Project,
%     Glasgow University, 1990-2000
%

% \documentstyle[preprint]{acmconf}
\documentclass[11pt]{article}
\oddsidemargin 0.1 in       %   Note that \oddsidemargin = \evensidemargin
\evensidemargin 0.1 in
\marginparwidth 0.85in    %   Narrow margins require narrower marginal notes
\marginparsep 0 in 
\sloppy

%\usepackage{epsfig}
\usepackage{shortvrb}
\MakeShortVerb{\@}

%\newcommand{\note}[1]{{\em Note: #1}}
\newcommand{\note}[1]{{{\bf Note:}\sl #1}}
\newcommand{\ToDo}[1]{{{\bf ToDo:}\sl #1}}
\newcommand{\Arg}[1]{\mbox{${\tt arg}_{#1}$}}
\newcommand{\bottom}{\perp}

\newcommand{\secref}[1]{Section~\ref{sec:#1}}
\newcommand{\figref}[1]{Figure~\ref{fig:#1}}
\newcommand{\Section}[2]{\section{#1}\label{sec:#2}}
\newcommand{\Subsection}[2]{\subsection{#1}\label{sec:#2}}
\newcommand{\Subsubsection}[2]{\subsubsection{#1}\label{sec:#2}}

% DIMENSION OF TEXT:
\textheight 8.5 in
\textwidth 6.25 in

\topmargin 0 in
\headheight 0 in
\headsep .25 in


\setlength{\parskip}{0.15cm}
\setlength{\parsep}{0.15cm}
\setlength{\topsep}{0cm}	% Reduces space before and after verbatim,
				% which is implemented using trivlist 
\setlength{\parindent}{0cm}

\renewcommand{\textfraction}{0.2}
\renewcommand{\floatpagefraction}{0.7}

\begin{document}

\title{The GHCi Draft Design, round 2}
\author{MSR Cambridge Haskell Crew \\
        Microsoft Research Ltd., Cambridge}

\maketitle

%%%\tableofcontents
%%%\newpage

%%-----------------------------------------------------------------%%
\section{Details}

\subsection{Outline of the design}
\label{sec:details-intro}

The design falls into three major parts:
\begin{itemize}
\item The compilation manager (CM), which coordinates the 
      system and supplies a HEP-like interface to clients.
\item The module compiler (@compile@), which translates individual
      modules to interpretable or machine code.
\item The linker (@link@),
      which maintains the executable image in interpreted mode.
\end{itemize}

There are also three auxiliary parts: the finder, which locates
source, object and interface files, the summariser, which quickly
finds dependency information for modules, and the static info
(compiler flags and package details), which is unchanged over the
course of a session.

This section continues with an overview of the session-lifetime data
structures.  Then follows the finder (section~\ref{sec:finder}),
summariser (section~\ref{sec:summariser}), 
static info (section~\ref{sec:staticinfo}),
and finally the three big sections
(\ref{sec:manager},~\ref{sec:compiler},~\ref{sec:linker})
on the compilation manager, compiler and linker respectively.

\subsubsection*{Some terminology}

Lifetimes: the phrase {\bf session lifetime} covers a complete run of
GHCI, encompassing multiple recompilation runs.  {\bf Module lifetime}
is a lot shorter, being that of data needed to translate a single
module, but then discarded, for example Core, AbstractC, Stix trees.

Data structures with module lifetime are well documented and understood.
This document is mostly concerned with session-lifetime data.
Most of these structures are ``owned'' by CM, since that's
the only major component of GHCI which deals with session-lifetime
issues. 

Modules and packages: {\bf home} refers to modules in this package,
precisely the ones tracked and updated by the compilation manager.
{\bf Package} refers to all other packages, which are assumed static.

\subsubsection*{A summary of all session-lifetime data structures}

These structures have session lifetime but not necessarily global
visibility.  Subsequent sections elaborate who can see what.
\begin{itemize}
\item {\bf Home Symbol Table (HST)} (owner: CM) holds the post-renaming
      environments created by compiling each home module.
\item {\bf Home Interface Table (HIT)} (owner: CM) holds in-memory
      representations of the interface file created by compiling 
      each home module.
\item {\bf Unlinked Images (UI)} (owner: CM) are executable but as-yet
      unlinked translations of home modules only.
\item {\bf Module Graph (MG)} (owner: CM) is the current module graph.
\item {\bf Static Info (SI)} (owner: CM) is the package configuration
      information (PCI) and compiler flags (FLAGS).
\item {\bf Persistent Compiler State (PCS)} (owner: @compile@)
      is @compile@'s private cache of information about package
      modules.
\item {\bf Persistent Linker State (PLS)} (owner: @link@) is
      @link@'s private information concerning the current 
      state of the (in-memory) executable image.
\end{itemize}


%%-- --- --- --- --- --- --- --- --- --- --- --- --- --- --- --- --%%
\subsection{The finder (\mbox{\tt type Finder})}
\label{sec:finder}

@Path@ could be an indication of a location in a filesystem, or it
could be some more generic kind of resource identifier, a URL for
example.
\begin{verbatim}
   data Path = ...
\end{verbatim}

And some names.  @Module@s are now used as primary keys for various
maps, so they are given a @Unique@.
\begin{verbatim}
   type ModName = String      -- a module name
   type PkgName = String      -- a package name
   type Module  = -- contains ModName and a Unique, at least
\end{verbatim}

A @ModLocation@ says where a module is, what it's called and in what
form it is.
\begin{verbatim}
   data ModLocation = SourceOnly Module Path         -- .hs
                    | ObjectCode Module Path Path    -- .o, .hi
                    | InPackage  Module PkgName
                          -- examine PCI to determine package Path
\end{verbatim}

The module finder generates @ModLocation@s from @ModName@s.  We expect
it will assume packages to be static, but we want to be able to track
changes in home modules during the session.  Specifically, we want to
be able to notice that a module's object and interface have been
updated, presumably by a compile run outside of the GHCI session.
Hence the two-stage type:
\begin{verbatim}
   type Finder = ModName -> IO ModLocation
   newFinder :: PCI -> IO Finder
\end{verbatim}
@newFinder@ examines the package information right at the start, but 
returns an @IO@-typed function which can inspect home module changes
later in the session.


%%-- --- --- --- --- --- --- --- --- --- --- --- --- --- --- --- --%%
\subsection{The summariser (\mbox{\tt summarise})}
\label{sec:summariser}

A @ModSummary@ records the minimum information needed to establish the
module graph and determine whose source has changed.  @ModSummary@s
can be created quickly.
\begin{verbatim}
   data ModSummary = ModSummary 
                        ModLocation   -- location and kind
                        (Maybe (String, Fingerprint))
                                      -- source and fingerprint if .hs
                        (Maybe [ModName])     -- imports if .hs or .hi

   type Fingerprint = ...  -- file timestamp, or source checksum?

   summarise :: ModLocation -> IO ModSummary
\end{verbatim}

The summary contains the location and source text, and the location
contains the name.  We would like to remove the assumption that
sources live on disk, but I'm not sure this is good enough yet.

\ToDo{Should @ModSummary@ contain source text for interface files too?}
\ToDo{Also say that @ModIFace@ contains its module's @ModSummary@  (why?).}


%%-- --- --- --- --- --- --- --- --- --- --- --- --- --- --- --- --%%
\subsection{Static information (SI)}
\label{sec:staticinfo}

PCI, the package configuration information, is a list of @PkgInfo@,
each containing at least the following:
\begin{verbatim}
   data PkgInfo
      = PkgInfo PkgName    -- my name
                Path       -- path to my base location
                [PkgName]  -- who I depend on
                [ModName]  -- modules I supply
                [Unlinked] -- paths to my object files

   type PCI = [PkgInfo]
\end{verbatim}
The @Path@s in it, including those in the @Unlinked@s, are set up
when GHCI starts.  

FLAGS is a bunch of compiler options.  We haven't figured out yet how
to partition them into those for the whole session vs those for
specific source files, so currently the best we can do is:
\begin{verbatim}
   data FLAGS = ...
\end{verbatim}

The static information (SI) is the both of these:
\begin{verbatim}
   data SI = SI PCI
                FLAGS
\end{verbatim}



%%-- --- --- --- --- --- --- --- --- --- --- --- --- --- --- --- --%%
\subsection{The Compilation Manager (CM)}
\label{sec:manager}

\subsubsection{Data structures owned by CM}

CM maintains two maps (HST, HIT) and a set (UI).  It's important to
realise that CM only knows about the map/set-ness, and has no idea
what a @ModDetails@, @ModIFace@ or @Linkable@ is.  Only @compile@ and
@link@ know that, and CM passes these types around without
inspecting them.

\begin{itemize}
\item
   {\bf Home Symbol Table (HST)} @:: FiniteMap Module ModDetails@

   The @ModDetails@ (a couple of layers down) contain tycons, classes,
   instances, etc, collectively known as ``entities''.  Referrals from
   other modules to these entities is direct, with no intervening
   indirections of any kind; conversely, these entities refer directly
   to other entities, regardless of module boundaries.  HST only holds
   information for home modules; the corresponding wired-up details
   for package (non-home) modules are created on demand in the package
   symbol table (PST) inside the persistent compiler's state (PCS).

   CM maintains the HST, which is passed to, but not modified by,
   @compile@.  If compilation of a module is successful, @compile@
   returns the resulting @ModDetails@ (inside the @CompResult@) which
   CM then adds to HST.

   CM throws away arbitrarily large parts of HST at the start of a
   rebuild, and uses @compile@ to incrementally reconstruct it.

\item
   {\bf Home Interface Table (HIT)} @:: FiniteMap Module ModIFace@

   (Completely private to CM; nobody else sees this).

   Compilation of a module always creates a @ModIFace@, which contains
   the unlinked symbol table entries.  CM maintains this @FiniteMap@
   @ModName@ @ModIFace@, with session lifetime.  CM never throws away
   @ModIFace@s, but it does update them, by passing old ones to
   @compile@ if they exist, and getting new ones back.

   CM acquires @ModuleIFace@s from @compile@, which it only applies
   to modules in the home package.  As a result, HIT only contains
   @ModuleIFace@s for modules in the home package.  Those from other
   packages reside in the package interface table (PIT) which is a
   component of PCS.

\item
   {\bf Unlinked Images (UI)} @:: Set Linkable@

   The @Linkable@s in UI represent executable but as-yet unlinked
   module translations.  A @Linkable@ can contain the name of an
   object, archive or DLL file.  In interactive mode, it may also be
   the STG trees derived from translating a module.  So @compile@
   returns a @Linkable@ from each successful run, namely that of
   translating the module at hand.  

   At link-time, CM supplies @Linkable@s for the upwards closure of
   all packages which have changed, to @link@.  It also examines the
   @ModSummary@s for all home modules, and by examining their imports
   and the SI.PCI (package configuration info) it can determine the
   @Linkable@s from all required imported packages too.

   @Linkable@s and @ModIFace@s have a close relationship.  Each
   translated module has a corresponding @Linkable@ somewhere.
   However, there may be @Linkable@s with no corresponding modules
   (the RTS, for example).  Conversely, multiple modules may share a
   single @Linkable@ -- as is the case for any module from a
   multi-module package.  For these reasons it seems appropriate to
   keep the two concepts distinct.  @Linkable@s also provide
   information about the sequence in which individual package
   components should be linked, and that isn't the business of any
   specific module to know.

   CM passes @compile@ a module's old @ModIFace@, if it has one, in
   the hope that the module won't need recompiling.  If so, @compile@
   can just return the new @ModDetails@ created from it, and CM will
   re-use the old @ModIFace@.  If the module {\em is} recompiled (or 
   scheduled to be loaded from disk), @compile@ returns both the 
   new @ModIFace@ and new @Linkable@.

\item 
   {\bf Module Graph (MG)} @:: known-only-to-CM@

   Records, for CM's purposes, the current module graph,
   up-to-dateness and summaries.  More details when I get to them.
   Only contains home modules.
\end{itemize}
Probably all this stuff is rolled together into the Persistent CM
State (PCMS):
\begin{verbatim}
  data PCMS = PCMS HST HIT UI MG
  emptyPCMS :: IO PCMS
\end{verbatim}

\subsubsection{What CM implements}
It pretty much implements the HEP interface.  First, though, define a 
containing structure for the state of the entire CM system and its
subsystems @compile@ and @link@:
\begin{verbatim}
   data CmState 
      = CmState PCMS      -- CM's stuff
                PCS       -- compile's stuff
                PLS       -- link's stuff
                SI        -- the static info, never changes
                Finder    -- the finder
\end{verbatim}

The @CmState@ is threaded through the HEP interface.  In reality
this might be done using @IORef@s, but for clarity:
\begin{verbatim}
  type ModHandle = ... (opaque to CM/HEP clients) ...
  type HValue    = ... (opaque to CM/HEP clients) ...

  cmInit       :: FLAGS 
               -> [PkgInfo]
               -> IO CmState

  cmLoadModule :: CmState 
               -> ModName 
               -> IO (CmState, Either [SDoc] ModHandle)

  cmGetExpr    :: ModHandle 
               -> CmState 
               -> String -> IO (CmState, Either [SDoc] HValue)

  cmRunExpr    :: HValue -> IO ()   -- don't need CmState here
\end{verbatim}
Almost all the huff and puff in this document pertains to @cmLoadModule@.


\subsubsection{Implementing \mbox{\tt cmInit}}
@cmInit@ creates an empty @CmState@ using @emptyPCMS@, @emptyPCS@,
@emptyPLS@, making SI from the supplied flags and package info, and 
by supplying the package info the @newFinder@.


\subsubsection{Implementing \mbox{\tt cmLoadModule}}

\begin{enumerate}
\item {\bf Downsweep:} using @finder@ and @summarise@, chase from 
      the given module to
      establish the new home module graph (MG).  Do not chase into
      package modules.
\item Remove from HIT, HST, UI any modules in the old MG which are
      not in the new one.  The old MG is then replaced by the new one.
\item Topologically sort MG to generate a bottom-to-top traversal
      order, giving a worklist.
\item {\bf Upsweep:} call @compile@ on each module in the worklist in 
      turn, passing it
      the ``correct'' HST, PCS, the old @ModIFace@ if
      available, and the summary.  ``Correct'' HST in the sense that
      HST contains only the modules in the this module's downward
      closure, so that @compile@ can construct the correct instance
      and rule environments simply as the union of those in 
      the module's downward closure.

      If @compile@ doesn't return a new interface/linkable pair,
      compilation wasn't necessary.  Either way, update HST with
      the new @ModDetails@, and UI and HIT respectively if a 
      compilation {\em did} occur.

      Keep going until the root module is successfully done, or
      compilation fails.
      
\item If the previous step terminated because compilation failed,
      define the successful set as those modules in successfully
      completed SCCs, i.e. all @Linkable@s returned by @compile@ excluding
      those from modules in any cycle which includes the module which failed.
      Remove from HST, HIT, UI and MG all modules mentioned in MG which 
      are not in the successful set.  Call @link@ with the successful
      set,
      which should succeed.  The net effect is to back off to a point
      in which those modules which are still aboard are correctly
      compiled and linked.

      If the previous step terminated successfully, 
      call @link@ passing it the @Linkable@s in the upward closure of
      all those modules for which @compile@ produced a new @Linkable@.
\end{enumerate}
As a small optimisation, do this:
\begin{enumerate}
\item[3a.] Remove from the worklist any module M where M's source
     hasn't changed and neither has the source of any module in M's
     downward closure.  This has the effect of not starting the upsweep
     right at the bottom of the graph when that's not needed.
     Source-change checking can be done quickly by CM by comparing
     summaries of modules in MG against corresponding 
     summaries from the old MG.
\end{enumerate}


%%-- --- --- --- --- --- --- --- --- --- --- --- --- --- --- --- --%%
\subsection{The compiler (\mbox{\tt compile})}
\label{sec:compiler}

\subsubsection{Data structures owned by \mbox{\tt compile}}

{\bf Persistent Compiler State (PCS)} @:: known-only-to-compile@

This contains info about foreign packages only, acting as a cache,
which is private to @compile@.  The cache never becomes out of
date.  There are three parts to it:

   \begin{itemize}
   \item
      {\bf Package Interface Table (PIT)} @:: FiniteMap Module ModIFace@

   @compile@ reads interfaces from modules in foreign packages, and
   caches them in the PIT.  Subsequent imports of the same module get
   them directly out of the PIT, avoiding slow lexing/parsing phases.
   Because foreign packages are assumed never to become out of date,
   all contents of PIT remain valid forever.  @compile@ of course
   tries to find package interfaces in PIT in preference to reading
   them from files.  

   Both successful and failed runs of @compile@ can add arbitrary
   numbers of new interfaces to the PIT.  The failed runs don't matter
   because we assume that packages are static, so the data cached even
   by a failed run is valid forever (ie for the rest of the session).

   \item
      {\bf Package Symbol Table (PST)} @:: FiniteMap Module ModDetails@

   Adding a package interface to PIT doesn't make it directly usable
   to @compile@, because it first needs to be wired (renamed +
   typechecked) into the spaghetti of the HST.  On the other hand,
   most modules only use a few entities from any imported interface,
   so wiring-in the interface at PIT-entry time might be a big time
   waster.  Also, wiring in an interface could mean reading other
   interfaces, and we don't want to do that unnecessarily.

   The PST avoids these problems by allowing incremental wiring-in to
   happen.  Pieces of foreign interfaces are copied out of the holding
   pen (HP), renamed, typechecked, and placed in the PST, but only as
   @compile@ discovers it needs them.  In the process of incremental
   renaming/typechecking, @compile@ may need to read more package
   interfaces, which are added to the PIT and hence to 
   HP.~\ToDo{How? When?}

   CM passes the PST to @compile@ and is returned an updated version
   on both success and failure.

   \item 
      {\bf Holding Pen (HP)} @:: HoldingPen@ 

   HP holds parsed but not-yet renamed-or-typechecked fragments of
   package interfaces.  As typechecking of other modules progresses,
   fragments are removed (``slurped'') from HP, renamed and
   typechecked, and placed in PCS.PST (see above).  Slurping a
   fragment may require new interfaces to be read into HP.  The hope
   is, though, that many fragments will never get slurped, reducing
   the total number of interfaces read (as compared to eager slurping).

   \end{itemize}

   PCS is opaque to CM; only @compile@ knows what's in it, and how to
   update it.  Because packages are assumed static, PCS never becomes
   out of date.  So CM only needs to be able to create an empty PCS,
   with @emptyPCS@, and thence just passes it through @compile@ with
   no further ado.

   In return, @compile@ must promise not to store in PCS any
   information pertaining to the home modules.  If it did so, CM would
   need to have a way to remove this information prior to commencing a
   rebuild, which conflicts with PCS's opaqueness to CM.




\subsubsection{What {\tt compile} does}
@compile@ is necessarily somewhat complex.  We've decided to do away
with private global variables -- they make the design specification
less clear, although the implementation might use them.  Without
further ado:
\begin{verbatim}
   compile :: SI          -- obvious
           -> Finder      -- to find modules
           -> ModSummary  -- summary, including source
           -> Maybe ModIFace
                          -- former summary, if avail
           -> HST         -- for home module ModDetails
           -> PCS         -- IN: the persistent compiler state

           -> IO CompResult

   data CompResult
      = CompOK  ModDetails   -- new details (== HST additions)
                (Maybe (ModIFace, Linkable))
                             -- summary and code; Nothing => compilation
                             -- not needed (old summary and code are still valid)
                PCS          -- updated PCS
                [SDoc]       -- warnings

      | CompErrs PCS         -- updated PCS
                 [SDoc]      -- warnings and errors

   data PCS
      = MkPCS PIT         -- package interfaces
              PST         -- post slurping global symtab contribs
              HoldingPen  -- pre slurping interface bits and pieces

   emptyPCS :: IO PCS     -- since CM has no other way to make one
\end{verbatim}
Although @compile@ is passed three of the global structures (FLAGS,
HST and PCS), it only modifies PCS.  The rest are modified by CM as it
sees fit, from the stuff returned in the @CompResult@.

@compile@ is allowed to return an updated PCS even if compilation
errors occur, since the information in it pertains only to foreign
packages and is assumed to be always-correct.

What @compile@ does: \ToDo{A bit vague ... needs refining.  How does
                           @finder@ come into the game?}
\begin{itemize}
\item Figure out if this module needs recompilation.
   \begin{itemize}
   \item If there's no old @ModIFace@, it does.  Else:
   \item Compare the @ModSummary@ supplied with that in the
         old @ModIFace@.  If the source has changed, recompilation
         is needed.  Else:
   \item Compare the usage version numbers in the old @ModIFace@ with
         those in the imported @ModIFace@s.  All needed interfaces
         for this should be in either HIT or PIT.  If any version
         numbers differ, recompilation is needed.
   \item Otherwise it isn't needed.   
   \end{itemize}

\item
   If recompilation is not needed, create a new @ModDetails@ from the
   old @ModIFace@, looking up information in HST and PCS.PST as
   necessary.  Return the new details, a @Nothing@ denoting
   compilation was not needed, the PCS \ToDo{I don't think the PCS
   should be updated, but who knows?}, and an empty warning list.

\item
   Otherwise, compilation is needed.  

   If the module is only available in object+interface form, read the
   interface, make up details, create a linkable pointing at the
   object code.  \ToDo{Does this involve reading any more interfaces?  Does
   it involve updating PST?}
   
   Otherwise, translate from source, then create and return: an
   details, interface, linkable, updated PST, and warnings.

   When looking for a new interface, search HST, then PCS.PIT, and only
   then read from disk.  In which case add the new interface(s) to
   PCS.PIT.  
   
   \ToDo{If compiling a module with a boot-interface file, check the 
   boot interface against the inferred interface.}
\end{itemize}


\subsubsection{Contents of \mbox{\tt ModDetails}, 
               \mbox{\tt ModIFace} and \mbox{\tt HoldingPen}}
Only @compile@ can see inside these three types -- they are opaque to
everyone else.  @ModDetails@ holds the post-renaming,
post-typechecking environment created by compiling a module.

\begin{verbatim}
   data ModDetails
      = ModDetails {
           moduleExports :: Avails
           moduleEnv     :: GlobalRdrEnv    -- == FM RdrName [Name]
           typeEnv       :: FM Name TyThing -- TyThing is in TcEnv.lhs
           instEnv       :: InstEnv
           fixityEnv     :: FM Name Fixity
           ruleEnv       :: FM Id [Rule]
        }
\end{verbatim}

@ModIFace@ is nearly the same as @ParsedIFace@ from @RnMonad.lhs@:
\begin{verbatim}
   type ModIFace = ParsedIFace    -- not really, but ...
   data ParsedIface
      = ParsedIface {
           pi_mod       :: Module,                   -- Complete with package info
           pi_vers      :: Version,                  -- Module version number
           pi_orphan    :: WhetherHasOrphans,        -- Whether this module has orphans
           pi_usages    :: [ImportVersion OccName],  -- Usages
           pi_exports   :: [ExportItem],             -- Exports
           pi_insts     :: [RdrNameInstDecl],        -- Local instance declarations
           pi_decls     :: [(Version, RdrNameHsDecl)],    -- Local definitions
           pi_fixity    :: (Version, [RdrNameFixitySig]), -- Local fixity declarations, 
                                                          -- with their version
           pi_rules     :: (Version, [RdrNameRuleDecl]),  -- Rules, with their version
           pi_deprecs   :: [RdrNameDeprecation]           -- Deprecations
       }
\end{verbatim}

@HoldingPen@ is a cleaned-up version of that found in @RnMonad.lhs@, 
retaining just the 3 pieces actually comprising the holding pen:
\begin{verbatim}
   data HoldingPen 
      = HoldingPen {
           iDecls :: DeclsMap,     -- A single, global map of Names to decls

           iInsts :: IfaceInsts,
           -- The as-yet un-slurped instance decls; this bag is depleted when we
           -- slurp an instance decl so that we don't slurp the same one twice.
           -- Each is 'gated' by the names that must be available before
           -- this instance decl is needed.

           iRules :: IfaceRules
           -- Similar to instance decls, only for rules
        }
\end{verbatim}

%%-- --- --- --- --- --- --- --- --- --- --- --- --- --- --- --- --%%
\subsection{The linker (\mbox{\tt link})}
\label{sec:linker}

\subsubsection{Data structures owned by the linker}

In the same way that @compile@ has a persistent compiler state (PCS),
the linker has a persistent (session-lifetime) state, PLS, the
Linker's Persistent State.  In batch mode PLS is entirely irrelevant,
because there is only a single link step, and can be a unit value
ignored by everybody.  In interactive mode PLS is composed of the
following three parts:

\begin{itemize}
\item 
\textbf{The Source Symbol Table (SST)}@ :: FiniteMap RdrName HValue@   
  The source symbol table is used when linking interpreted code.
  Unlinked interpreted code consists of an STG  tree where
  the leaves are @RdrNames@.  The linker's job is to resolve these to
  actual addresses (the alternative is to resolve these lazily when
  the code is run, but this requires passing the full symbol table
  through the interpreter and the repeated lookups will probably be
  expensive).

  The source symbol table therefore maps @RdrName@s to @HValue@s, for
  every @RdrName@ that currently \emph{has} an @HValue@, including all
  exported functions from object code modules that are currently
  linked in.  Linking therefore turns a @StgTree RdrName@ into an
  @StgTree HValue@.

  It is important that we can prune this symbol table by throwing away
  the mappings for an entire module, whenever we recompile/relink a
  given module.  The representation is therefore probably a two-level
  mapping, from module names, to function/constructor names, to
  @HValue@s.

\item \textbf{The Object Symbol Table (OST)}@ :: FiniteMap String Addr@
  This is a lower level symbol table, mapping symbol names in object
  modules to their addresses in memory.  It is used only when
  resolving the external references in an object module, and contains
  only entries that are defined in object modules.

  Why have two symbol tables?  Well, there is a clear distinction
  between the two: the source symbol table maps Haskell symbols to
  Haskell values, and the object symbol table maps object symbols to
  addresses.  There is some overlap, in that Haskell symbols certainly
  have addresses, and we could look up a Haskell symbol's address by
  manufacturing the right object symbol and looking that up in the
  object symbol table, but this is likely to be slow and would force
  us to extend the object symbol table with all the symbols
  ``exported'' by interpreted code.  Doing it this way enables us to
  decouple the object management subsystem from the rest of the linker
  with a minimal interface; something like

  \begin{verbatim}
  loadObject   :: Unlinked -> IO Object
  unloadModule :: Unlinked -> IO ()
  lookupSymbol :: String   -> IO Addr
  \end{verbatim}

  Rather unfortunately we need @lookupSymbol@ in order to populate the
  source symbol table when linking in a new compiled module.  Our
  object management subsystem is currently written in C, so decoupling
  this interface as much as possible is highly desirable.

\item
   {\bf Linked Image (LI)} @:: no-explicit-representation@

   LI isn't explicitly represented in the system, but we record it
   here for completeness anyway.  LI is the current set of
   linked-together module, package and other library fragments
   constituting the current executable mass.  LI comprises:
   \begin{itemize}
   \item Machine code (@.o@, @.a@, @.DLL@ file images) in memory.
         These are loaded from disk when needed, and stored in
         @malloc@ville.  To simplify storage management, they are
         never freed or reused, since this creates serious
         complications for storage management.  When no longer needed,
         they are simply abandoned.  New linkings of the same object
         code produces new copies in memory.  We hope this not to be
         too much of a space leak.
   \item STG trees, which live in the GHCI heap and are managed by the
         storage manager in the usual way.  They are held alive (are 
         reachable) via the @HValue@s in the OST.  Such @HValue@s are
         applications of the interpreter function to the trees
         themselves.  Linking a tree comprises travelling over the 
         tree, replacing all the @Id@s with pointers directly to the
         relevant @_closure@ labels, as determined by searching the
         OST.  Once the leaves are linked, trees are wrapped with the
         interpreter function.  The resulting @HValue@s then behave
         indistinguishably from compiled versions of the same code.
   \end{itemize}
   Because object code is outside the heap and never deallocated,
   whilst interpreted code is held alive via the HST, there's no need
   to have a data structure which ``is'' the linked image.

   For batch compilation, LI doesn't exist because OST doesn't exist,
   and because @link@ doesn't load code into memory, instead just
   invokes the system linker.

   \ToDo{Do we need to say anything about CAFs and SRTs?  Probably ...}
\end{itemize}
As with PCS, CM has no way to create an initial PLS, so we supply
@emptyPLS@ for that purpose.

\subsubsection{The linker's interface}

In practice, the PLS might be hidden in the I/O monad rather
than passed around explicitly.  (The same might be true for PCS).
Anyway:

\begin{verbatim}
   data PLS -- as described above; opaque to everybody except the linker

   link :: PCI -> ??? -> [[Linkable]] -> PLS -> IO LinkResult

   data LinkResult = LinkOK   PLS
                   | LinkErrs PLS [SDoc]

   emptyPLS :: IO PLS     -- since CM has no other way to make one
\end{verbatim}

CM uses @link@ as follows:

After repeatedly using @compile@ to compile all modules which are
out-of-date, the @link@ is invoked.  The @[[Linkable]]@ argument to
@link@ represents the list of (recursive groups of) home modules which
have been newly compiled, along with @Linkable@s for each of
the packages in use (the compilation manager knows which external
packages are referenced by the home package).  The order of the list
is important: it is sorted in such a way that linking any prefix of
the list will result in an image with no unresolved references.  Note
that for batch linking there may be further restrictions; for example
it may not be possible to link recursive groups containing libraries.

@link@ does the following:

\begin{itemize}
  \item 
  In batch mode, do nothing.  In interactive mode,
  examine the supplied @[[Linkable]]@ to determine which home 
  module @Unlinked@s are new.  Remove precisely these @Linkable@s 
  from PLS.  (In fact we really need to remove their upwards
  transitive closure, but I think it is an invariant that CM will
  supply an upwards transitive closure of new modules).
  See below for descriptions of @Linkable@ and @Unlinked@.

  \item 
  Batch system: invoke the external linker to link everything in one go.
  Interactive: bind the @Unlinked@s for the newly compiled modules,
  plus those for any newly required packages, into PLS.

  Note that it is the linker's responsibility to remember which
  objects and packages have already been linked.  By comparing this
  with the @Linkable@s supplied to @link@, it can determine which
  of the linkables in LI are out of date
\end{itemize}

If linking in of a group should fail for some reason, @link@ should
not modify its PLS at all.  In other words, linking each group
is atomic; it either succeeds or fails.

\subsubsection*{\mbox{\tt Unlinked} and \mbox{\tt Linkable}}

Two important types: @Unlinked@ and @Linkable@.  The latter is a 
higher-level representation involving multiple of the former.
An @Unlinked@ is a reference to unlinked executable code, something
a linker could take as input:

\begin{verbatim}
   data Unlinked = DotO   Path
                 | DotA   Path            
                 | DotDLL Path
                 | Trees  [StgTree RdrName]
\end{verbatim}

The first three describe the location of a file (presumably)
containing the code to link.  @Trees@, which only exists in
interactive mode, gives a list of @StgTrees@, in which the unresolved
references are @RdrNames@ -- hence it's non-linkedness.  Once linked,
those @RdrNames@ are replaced with pointers to the machine code
implementing them.

A @Linkable@ gathers together several @Unlinked@s and associates them
with either a module or package:

\begin{verbatim}
   data Linkable = LM Module  [Unlinked]   -- a module
                 | LP PkgName              -- a package
\end{verbatim}

The order of the @Unlinked@s in the list is important, as
they are linked in left-to-right order.  The @Unlinked@ objects for a
particular package can be obtained from the package configuration (see
Section \ref{sec:staticinfo}).

\ToDo{When adding @Addr@s from an object module to SST, we need to
      somehow find out the @RdrName@s of the symbols exported by that
      module. 
      So we'd need to pass in the @ModDetails@ or @ModIFace@ or some such?}



%%-----------------------------------------------------------------%%
\section{Background ideas}
\subsubsection*{Out of date, but correct in spirit}

\subsection{Restructuring the system}

At the moment @hsc@ compiles one source module into C or assembly.
This functionality is pushed inside a function called @compile@,
introduced shortly.  The main new chunk of code is CM, the compilation manager,
which supervises multiple runs of @compile@ so as to create up-to-date
translations of a whole bunch of modules, as quickly as possible.
CM also employs some minor helper functions, @finder@, @summarise@ and
@link@, to do its work.

Our intent is to allow CM to be used as the basis either of a 
multi-module, batch mode compilation system, or to supply an
interactive environment similar to that of Hugs.
Only minor modifications to the behaviour of @compile@ and @link@ 
are needed to give these different behaviours.

CM and @compile@, and, for interactive use, an interpreter, are the
main code components.  The most important data structure is the global
symbol table; much design effort has been expended thereupon.


\subsection{How the global symbol table is implemented}

The top level symbol table is a @FiniteMap@ @ModuleName@
@ModuleDetails@.  @ModuleDetails@ contains essentially the environment
created by compiling a module.  CM manages this finite map, adding and
deleting module entries as required.

The @ModuleDetails@ for a module @M@ contains descriptions of all
tycons, classes, instances, values, unfoldings, etc (henceforth
referred to as ``entities''), available from @M@.  These are just
trees in the GHCI heap.  References from other modules to these
entities is direct -- when you have a @TyCon@ in your hand, you really
have a pointer directly to the @TyCon@ structure in the defining module,
rather than some kind of index into a global symbol table.  So there
is a global symbol table, but it has a distributed (spaghetti-like?)
nature.

This gives fast and convenient access to tycon, class, instance,
etc, information.  But because there are no levels of indirection,
there's a problem when we replace @M@ with an updated version of @M@.
We then need to find all references to entities in the old @M@'s
spaghetti, and replace them with pointers to the new @M@'s spaghetti.
This problem motivates a large part of the design.



\subsection{Implementing incremental recompilation -- simple version}
Given the following module graph
\begin{verbatim}
         D
       /   \
      /     \
     B       C
      \     /
       \   /
         A
\end{verbatim}
(@D@ imports @B@ and @C@, @B@ imports @A@, @C@ imports @A@) the aim is to do the
least possible amount of compilation to bring @D@ back up to date.  The
simplest scheme we can think of is:
\begin{itemize}
\item {\bf Downsweep}: 
  starting with @D@, re-establish what the current module graph is
  (it might have changed since last time).  This means getting a
  @ModuleSummary@ of @D@.  The summary can be quickly generated,
  contains @D@'s import lists, and gives some way of knowing whether
  @D@'s source has changed since the last time it was summarised.

  Transitively follow summaries from @D@, thereby establishing the
  module graph.
\item
  Remove from the global symbol table (the @FiniteMap@ @ModuleName@
  @ModuleDetails@) the upwards closure of all modules in this package
  which are out-of-date with respect to their previous versions.  Also
  remove all modules no longer reachable from @D@.
\item {\bf Upsweep}:
  Starting at the lowest point in the still-in-date module graph,
  start compiling upwards, towards @D@.  At each module, call
  @compile@, passing it a @FiniteMap@ @ModuleName@ @ModuleDetails@,
  and getting a new @ModuleDetails@ for the module, which is added to
  the map.

  When compiling a module, the compiler must be able to know which
  entries in the map are for modules in its strict downwards closure,
  and which aren't, so that it can manufacture the instance
  environment correctly (as union of instances in its downwards
  closure).
\item
  Once @D@ has been compiled, invoke some kind of linking phase
  if batch compilation.  For interactive use, can either do it all
  at the end, or as you go along.
\end{itemize}
In this simple world, recompilation visits the upwards closure of
all changed modules.  That means when a module @M@ is recompiled,
we can be sure no-one has any references to entities in the old @M@,
because modules importing @M@ will have already been removed from the 
top-level finite map in the second step above.

The upshot is that we don't need to worry about updating links to @M@ in
the global symbol table -- there shouldn't be any to update.
\ToDo{What about mutually recursive modules?}

CM will happily chase through module interfaces in other packages in
the downsweep.  But it will only process modules in this package
during the upsweep.  So it assumes that modules in other packages
never become out of date.  This is a design decision -- we could have
decided otherwise.

In fact we go further, and require other packages to be compiled,
i.e. to consist of a collection of interface files, and one or more
source files.  CM will never apply @compile@ to a foreign package
module, so there's no way a package can be built on the fly from source.

We require @compile@ to cache foreign package interfaces it reads, so
that subsequent uses don't have to re-read them.  The cache never
becomes out of date, since we've assumed that the source of foreign
packages doesn't change during the course of a session (run of GHCI).
As well as caching interfaces, @compile@ must cache, in some sense,
the linkable code for modules.  In batch compilation this might simply
mean remembering the names of object files to link, whereas in
interactive mode @compile@ probably needs to load object code into
memory in preparation for in-memory linking.

Important signatures for this simple scheme are:
\begin{verbatim}
   finder :: ModuleName -> ModLocation

   summarise :: ModLocation -> IO ModSummary

   compile :: ModSummary 
              -> FM ModName ModDetails
              -> IO CompileResult

   data CompileResult = CompOK  ModDetails
                      | CompErr [ErrMsg]

   link :: [ModLocation] -> [PackageLocation] -> IO Bool  -- linked ok?
\end{verbatim}


\subsection{Implementing incremental recompilation -- clever version}

So far, our upsweep, which is the computationally expensive bit,
recompiles a module if either its source is out of date, or it 
imports a module which has been recompiled.  Sometimes we know
we can do better than this:
\begin{verbatim}
   module B where                module A 
   import A ( f )                {-# NOINLINE f #-}
   ... f ...                     f x = x + 42
\end{verbatim}
If the definition of @f@ is changed to @f x = x + 43@, the simple
upsweep would recompile @B@ unnecessarily.  We would like to detect
this situation and avoid propagating recompilation all the way to the
top.  There are two parts to this: detecting when a module doesn't
need recompilation, and managing inter-module references in the
global symbol table.

\subsubsection*{Detecting when a module doesn't need recompilation}

To do this, we introduce a new concept: the @ModuleIFace@.  This is
effectively an in-memory interface file.  References to entities in
other modules are done via strings, rather than being pointers
directly to those entities.  Recall that, by comparison,
@ModuleDetails@ do contain pointers directly to the entities they
refer to.  So a @ModuleIFace@ is not part of the global symbol table.

As before, compiling a module produces a @ModuleDetails@ (inside the
@CompileResult@), but it also produces a @ModuleIFace@.  The latter
records, amongst things, the version numbers of all imported entities
needed for the compilation of that module.  @compile@ optionally also
takes the old @ModuleIFace@ as input during compilation:
\begin{verbatim}
   data CompileResult = CompOK  ModDetails ModIFace
                      | CompErr [ErrMsg]

   compile :: ModSummary 
              -> FM ModName ModDetails
              -> Maybe ModuleIFace
              -> IO CompileResult
\end{verbatim}
Now, if the @ModuleSummary@ indicates this module's source hasn't
changed, we only need to recompile it if something it depends on has
changed.  @compile@ can detect this by inspecting the imported entity
version numbers in the module's old @ModuleIFace@, and comparing them
with the version numbers from the entities in the modules being
imported.  If they are all the same, nothing it depends on has
changed, so there's no point in recompiling.

\subsubsection*{Managing inter-module references in the global symbol table}

In the above example with @A@, @B@ and @f@, the specified change to @f@ would
require @A@ but not @B@ to be recompiled.  That generates a new
@ModuleDetails@ for @A@.  Problem is, if we leave @B@'s @ModuleDetails@ 
unchanged, they continue to refer (directly) to the @f@ in @A@'s old
@ModuleDetails@.  This is not good, especially if equality between
entities is implemented using pointer equality.

One solution is to throw away @B@'s @ModuleDetails@ and recompile @B@.
But this is precisely what we're trying to avoid, as it's expensive.
Instead, a cheaper mechanism achieves the same thing: recreate @B@'s
details directly from the old @ModuleIFace@.  The @ModuleIFace@ will
(textually) mention @f@; @compile@ can then find a pointer to the 
up-to-date global symbol table entry for @f@, and place that pointer
in @B@'s @ModuleDetails@.  The @ModuleDetails@ are, therefore,
regenerated just by a quick lookup pass over the module's former
@ModuleIFace@.  All this applies, of course, only when @compile@ has
concluded it doesn't need to recompile @B@.

Now @compile@'s signature becomes a little clearer.  @compile@ has to
recompile the module, generating a fresh @ModuleDetails@ and
@ModuleIFace@, if any of the following hold:
\begin{itemize}
\item
  The old @ModuleIFace@ wasn't supplied, for some reason (perhaps
  we've never compiled this module before?)
\item
  The module's source has changed.
\item
  The module's source hasn't changed, but inspection of @ModuleIFaces@ 
  for this and its imports indicates that an imported entity has
  changed.
\end{itemize}
If none of those are true, we're in luck: quickly knock up a new
@ModuleDetails@ from the old @ModuleIFace@, and return them both.

As a result, the upsweep still visits all modules in the upwards
closure of those whose sources have changed.  However, at some point
we hopefully make a transition from generating new @ModuleDetails@ the
expensive way (recompilation) to a cheap way (recycling old
@ModuleIFaces@).  Either way, all modules still get new
@ModuleDetails@, so the global symbol table is correctly
reconstructed.


\subsection{How linking works, roughly}

When @compile@ translates a module, it produces a @ModuleDetails@,
@ModuleIFace@ and a @Linkable@.  The @Linkable@ contains the
translated but un-linked code for the module.  And when @compile@
ventures into an interface in package it hasn't seen so far, it
copies the package's object code into memory, producing one or more
@Linkable@s.  CM keeps track of these linkables.  

Once all modules have been @compile@d, CM invokes @link@, supplying
the all the @Linkable@s it knows about.  If @compile@ had also been
linking incrementally as it went along, @link@ doesn't have to do
anything.  On the other hand, @compile@ could choose not to be
incremental, and leave @link@ to do all the work.

@Linkable@s are opaque to CM.  For batch compilation, a @Linkable@
can record just the name of an object file, DLL, archive, or whatever,
in which case the CM's call to @link@ supplies exactly the set of
file names to be linked.  @link@ can pass these verbatim to the
standard system linker.




%%-----------------------------------------------------------------%%
\section{Ancient stuff}
\subsubsection*{Should be selectively merged into ``Background ideas''}

\subsection{Overall}
Top level structure is:
\begin{itemize}
\item The Compilation Manager (CM) calculates and maintains module
      dependencies, and knows how create up-to-date object or bytecode
      for a given module.  In doing so it may need to recompile 
      arbitrary other modules, based on its knowledge of the module
      dependencies.  
\item On top of the CM are the ``user-level'' services.  We envisage
      both a HEP-like interface, for interactive use, and an
      @hmake@ style batch compiler facility.
\item The CM only deals with inter-module issues.  It knows nothing
      about how to recompile an individual module, nor where the compiled
      result for a module lives, nor how to tell if 
      a module is up to date, nor how to find the dependencies of a module.
      Instead, these services are supplied abstractly to CM via a
      @Compiler@ record.  To a first approximation, a @Compiler@
      contains
      the same functionality as @hsc@ has had until now -- the ability to
      translate a single Haskell module to C/assembly/object/bytecode.

      Different clients of CM (HEP vs @hmake@) may supply different
      @Compiler@s, since they need slightly different behaviours.
      Specifically, HEP needs a @Compiler@ which creates bytecode
      in memory, and knows how to link it, whereas @hmake@ wants
      the traditional behaviour of emitting assembly code to disk,
      and making no attempt at linkage.
\end{itemize}

\subsection{Open questions}
\begin{itemize}
\item
  Error reporting from @open@ and @compile@.
\item
  Instance environment management
\item
  We probably need to make interface files say what
  packages they depend on (so that we can figure out
  which packages to load/link).
\item 
  CM is parameterised both by the client uses and the @Compiler@
  supplied.  But it doesn't make sense to have a HEP-style client
  attached to a @hmake@-style @Compiler@.  So, really, the 
  parameterising entity should contain both aspects, not just the
  current @Compiler@ contents.
\end{itemize}

\subsection{Assumptions}

\begin{itemize}
\item Packages other than the "current" one are assumed to be 
  already compiled.  
\item
  The "current" package is usually "MAIN",
  but we can set it with a command-line flag.
  One invocation of ghci has only one "current" package.
\item
  Packages are not mutually recursive
\item
  All the object code for a package P is in libP.a or libP.dll
\end{itemize}

\subsection{Stuff we need to be able to do}
\begin{itemize}
\item Create the environment in which a module has been translated,
      so that interactive queries can be satisfied as if ``in'' that
      module.
\end{itemize}

%%-----------------------------------------------------------------%%
\section{The Compilation Manager}

CM (@compilationManager@) is a functor, thus:
\begin{verbatim}
compilationManager :: Compiler -> IO HEP  -- IO so that it can create 
                                          -- global vars (IORefs)

data HEP = HEP {
        load          :: ModuleName -> IO (),
        compileString :: ModuleName -> String -> IO HValue,
        ....
   }

newCompiler :: IO Compiler   -- ??? this is a peer of compilationManager?

run :: HValue -> IO ()       -- Run an HValue of type IO ()
                             -- In HEP?
\end{verbatim}

@load@ is the central action of CM: its job is to bring a module and
all its descendents into an executable state, by doing the following:
\begin{enumerate}
\item 
   Use @summarise@ to descend the module hierarchy, starting from the
   nominated root, creating @ModuleSummary@s, and
   building a map @ModuleName@ @->@ @ModuleSummary@.  @summarise@ 
   expects to be passed absolute paths to files.  Use @finder@ to 
   convert module names to file paths.
\item
   Topologically sort the map, 
   using dependency info in the @ModuleSummary@s.
\item
   Clean up the symbol table by deleting the upward closure of 
   changed modules.
\item 
   Working bottom to top, call @compile@ on the upward closure of 
   all modules whose source has changed.  A module's source has
   changed when @sourceHasChanged@ indicates there is a difference
   between old and new summaries for the module.  Update the running
   @FiniteMap@ @ModuleName@ @ModuleDetails@ with the new details
   for this module.  Ditto for the running
   @FiniteMap@ @ModuleName@ @ModuleIFace@.
\item
   Call @compileDone@ to signify that we've reached the top, so
   that the batch system can now link.
\end{enumerate}


%%-----------------------------------------------------------------%%
\section{A compiler}

Most of the system's complexity is hidden inside the functions
supplied in the @Compiler@ record:
\begin{verbatim}        
data Compiler = Compiler {        

        finder :: PackageConf -> [Path] -> IO (ModuleName -> ModuleLocation)

        summarise :: ModuleLocation -> IO ModuleSummary

        compile :: ModuleSummary
                -> Maybe ModuleIFace 
                -> FiniteMap ModuleName ModuleDetails
                -> IO CompileResult

        compileDone     :: IO ()
        compileStarting :: IO ()   -- still needed?  I don't think so.
    }

type ModuleName = String (or some such)
type Path = String  -- an absolute file name
\end{verbatim}

\subsection{The module \mbox{\tt finder}}
The @finder@, given a package configuration file and a list of
directories to look in, will map module names to @ModuleLocation@s,
in which the @Path@s are filenames, probably with an absolute path
to them.
\begin{verbatim}
data ModuleLocation = SourceOnly Path        -- .hs
                    | ObjectCode Path Path   -- .o & .hi
                    | InPackage  Path        -- .hi
\end{verbatim}
@SourceOnly@ and @ObjectCode@ are unremarkable.  For sanity,
we require that a module's object and interface be in the same
directory.  @InPackage@ indicates that the module is in a 
different package.

@Module@ values -- perhaps all @Name@ish things -- contain the name of
their package.  That's so that 
\begin{itemize}
\item Correct code can be generated for in-DLL vs out-of-DLL refs.
\item We don't have version number dependencies for symbols
      imported from different packages.
\end{itemize}

Somehow or other, it will be possible to know all the packages
required, so that the for the linker can load them.
We could detect package dependencies by recording them in the
@compile@r's @ModuleIFace@ cache, and with that and the 
package config info, figure out the complete set of packages
to link.  Or look at the command line args on startup.

\ToDo{Need some way to tell incremental linkers about packages,
      since in general we'll need to load and link them before
      linking any modules in the current package.}


\subsection{The module \mbox{\tt summarise}r}
Given a filename of a module (\ToDo{presumably source or iface}),
create a summary of it.  A @ModuleSummary@ should contain only enough
information for CM to construct an up-to-date picture of the
dependency graph.  Rather than expose CM to details of timestamps,
etc, @summarise@ merely provides an up-to-date summary of any module.
CM can extract the list of dependencies from a @ModuleSummary@, but
other than that has no idea what's inside it.
\begin{verbatim}
data ModuleSummary = ... (abstract) ...

depsFromSummary :: ModuleSummary -> [ModuleName]   -- module names imported
sourceHasChanged :: ModuleSummary -> ModuleSummary -> Bool
\end{verbatim}
@summarise@ is intended to be fast -- a @stat@ of the source or
interface to see if it has changed, and, if so, a quick semi-parse to
determine the new imports.

\subsection{The module \mbox{\tt compile}r}
@compile@ traffics in @ModuleIFace@s and @ModuleDetails@.  

A @ModuleIFace@ is an in-memory representation of the contents of an
interface file, including version numbers, unfoldings and pragmas, and
the linkable code for the module.  @ModuleIFace@s are un-renamed,
using @HsSym@/@RdrNames@ rather than (globally distinct) @Names@.

@ModuleDetails@, by contrast, is an in-memory representation of the
static environment created by compiling a module.  It is phrased in
terms of post-renaming @Names@, @TyCon@s, etc, so it's basically a
renamed-to-global-uniqueness rendition of a @ModuleIFace@.

In an interactive session, we'll want to be able to evaluate
expressions as if they had been compiled in the scope of some
specified module.  This means that the @ModuleDetails@ must contain
the type of everything defined in the module, rather than just the
types of exported stuff.  As a consequence, @ModuleIFace@ must also
contain the type of everything, because it should always be possible
to generate a module's @ModuleDetails@ from its @ModuleIFace@.

CM maintains two mappings, one from @ModuleName@s to @ModuleIFace@s,
the other from @ModuleName@s to @ModuleDetail@s.  It passes the former
to each call of @compile@.  This is used to supply information about
modules compiled prior to this one (lower down in the graph).  The
returned @CompileResult@ supplies a new @ModuleDetails@ for the module
if compilation succeeded, and CM adds this to the mapping.  The
@CompileResult@ also supplies a new @ModuleIFace@, which is either the
same as that supplied to @compile@, if @compile@ decided not to
retranslate the module, or is the result of a fresh translation (from
source).  So these mappings are an explicitly-passed-around part of
the global system state.

@compile@ may also {\em optionally} also accumulate @ModuleIFace@s for
modules in different packages -- that is, interfaces which we read,
but never attempt to recompile source for.  Such interfaces, being
from foreign packages, never change, so @compile@ can accumulate them
in perpetuity in a private global variable.  Indeed, a major motivator
of this design is to facilitate this caching of interface files,
reading of which is a serious bottleneck for the current compiler.

When CM restarts compilation down at the bottom of the module graph,
it first needs to throw away all \ToDo{all?} @ModuleDetails@ in the
upward closure of the out-of-date modules.  So @ModuleDetails@ don't
persist across recompilations.  But @ModuleIFace@s do, since they
are conceptually equivalent to interface files.


\subsubsection*{What @compile@ returns}
@compile@ returns a @CompileResult@ to CM.
Note that the @compile@'s foreign-package interface cache can
become augmented even as a result of reading interfaces for a
compilation attempt which ultimately fails, although it will not be
augmented with a new @ModuleIFace@ for the failed module.
\begin{verbatim}
-- CompileResult is not abstract to the Compilation Manager
data CompileResult
   = CompOK   ModuleIFace 
              ModuleDetails    -- compiled ok, here are new details
                               -- and new iface

   | CompErr  [SDoc]           -- compilation gave errors

   | NoChange                  -- no change required, meaning:
                               -- exports, unfoldings, strictness, etc,
                               -- unchanged, and executable code unchanged
\end{verbatim}



\subsubsection*{Re-establishing local-to-global name mappings}
Consider
\begin{verbatim}
module Upper where                         module Lower ( f ) where
import Lower ( f )                         f = ...
g = ... f ...
\end{verbatim}
When @Lower@ is first compiled, @f@ is allocated a @Unique@
(presumably inside an @Id@ or @Name@?).  When @Upper@ is then
compiled, its reference to @f@ is attached directly to the
@Id@ created when compiling @Lower@.

If the definition of @f@ is now changed, but not the type,
unfolding, strictness, or any other thing which affects the way
it should be called, we will have to recompile @Lower@, but not
@Upper@.  This creates a problem -- @g@ will then refer to the
the old @Id@ for @f@, not the new one.  This may or may not
matter, but it seems safer to ensure that all @Unique@-based
references into child modules are always up to date.

So @compile@ recreates the @ModuleDetails@ for @Upper@ from 
the @ModuleIFace@ of @Upper@ and the @ModuleDetails@ of @Lower@.

The rule is: if a module is up to date with respect to its
source, but a child @C@ has changed, then either:
\begin{itemize}
\item On examination of the version numbers in @C@'s
      interface/@ModuleIFace@ that we used last time, we discover that
      an @Id@/@TyCon@/class/instance we depend on has changed.  So 
      we need to retranslate the module from its source, generating
      a new @ModuleIFace@ and @ModuleDetails@.
\item Or: there's nothing in @C@'s interface that we depend on.
      So we quickly recreate a new @ModuleDetails@ from the existing
      @ModuleIFace@, creating fresh links to the new @Unique@-world
      entities in @C@'s new @ModuleDetails@.
\end{itemize}

Upshot: we need to redo @compile@ on all modules all the way up,
rather than just the ones that need retranslation.  However, we hope
that most modules won't need retranslation -- just regeneration of the
@ModuleDetails@ from the @ModuleIFace@.  In effect, the @ModuleIFace@
is a quickly-compilable representation of the module's contents, just
enough to create the @ModuleDetails@.

\ToDo{Is there anything in @ModuleDetails@ which can't be
      recreated from @ModuleIFace@ ?}

So the @ModuleIFace@s persist across calls to @HEP.load@, whereas
@ModuleDetails@ are reconstructed on every compilation pass.  This
means that @ModuleIFace@s have the same lifetime as the byte/object
code, and so should somehow contain their code.

The behind-the-scenes @ModuleIFace@ cache has some kind of holding-pen
arrangement, to lazify the copying-out of stuff from it, and thus to
minimise redundant interface reading.  \ToDo{Burble burble.  More
details.}.

When CM starts working back up the module graph with @compile@, it
needs to remove from the travelling @FiniteMap@ @ModuleName@
@ModuleDetails@ the details for all modules in the upward closure of
the compilation start points.  However, since we're going to visit
precisely those modules and no others on the way back up, we might as
well just zap them the old @ModuleDetails@ incrementally.  This does
mean that the @FiniteMap@ @ModuleName@ @ModuleDetails@ will be
inconsistent until we reach the top.

In interactive mode, each @compile@ call on a module for which no
object code is available, or for which it is out of date wrt source,
emit bytecode into memory, update the resulting @ModuleIFace@ with the
address of the bytecode image, and link the image.

In batch mode, emit assembly or object code onto disk.  Record
somewhere \ToDo{where?} that this object file needs to go into the
final link.

When we reach the top, @compileDone@ is called, to signify that batch
linking can now proceed, if need be.

Modules in other packages never get a @ModuleIFace@ or @ModuleDetails@
entry in CM's maps -- those maps are only for modules in this package.
As previously mentioned, @compile@ may optionally cache @ModuleIFace@s
for foreign package modules.  When reading such an interface, we don't
need to read the version info for individual symbols, since foreign
packages are assumed static.

\subsubsection*{What's in a \mbox{\tt ModuleIFace}?}

Current interface file contents?


\subsubsection*{What's in a \mbox{\tt ModuleDetails}?}

There is no global symbol table @:: Name -> ???@.  To look up a
@Name@, first extract the @ModuleName@ from it, look that up in
the passed-in @FiniteMap@ @ModuleName@ @ModuleDetails@, 
and finally look in the relevant @Env@.

\ToDo{Do we still have the @HoldingPen@, or is it now composed from
per-module bits too?}
\begin{verbatim}
data ModuleDetails = ModuleDetails {

        moduleExports :: what it exports (Names)
                         -- roughly a subset of the .hi file contents

        moduleEnv     :: RdrName -> Name
                         -- maps top-level entities in this module to
                         -- globally distinct (Uniq-ified) Names
  
        moduleDefs    :: Bag Name -- All the things in the global symbol table
                                  -- defined by this module

        package       :: Package -- what package am I in?

        lastCompile   :: Date -- of last compilation

        instEnv       :: InstEnv                 -- local inst env
        typeEnv       :: Name -> TyThing         -- local tycon env?
   }

-- A (globally unique) symbol table entry. Note that Ids contain
-- unfoldings. 
data TyThing = AClass Class
             | ATyCon TyCon
             | AnId Id 
\end{verbatim}
What's the stuff in @ModuleDetails@ used for?
\begin{itemize}
\item @moduleExports@ so that the stuff which is visible from outside
      the module can be calculated.
\item @moduleEnv@: \ToDo{umm err}
\item @moduleDefs@: one reason we want this is so that we can nuke the
      global symbol table contribs from this module when it leaves the
      system.  \ToDo{except ... we don't have a global symbol table any
      more.}
\item @package@: we will need to chase arbitrarily deep into the
      interfaces of other packages.  Of course we don't want to 
      recompile those, but as we've read their interfaces, we may
      as well cache that info.  So @package@ indicates whether this
      module is in the default package, or, if not, which it is in.

      Also, when we come to linking, we'll need to know which
      packages are demanded, so we know to load their objects.

\item @lastCompile@: When the module was last compiled.  If the 
      source is older than that, then a recompilation can only be
      required if children have changed.
\item @typeEnv@: obvious??
\item @instEnv@: the instances contributed by this module only.  The
      Report allegedly says that when a module is translated, the
      available
      instance env is all the instances in the downward closure of
      itself in the module graph.
      
      We choose to use this simple representation -- each module 
      holds just its own instances -- and do the naive thing when
      creating an inst env for compilation with.  If this turns out
      to be a performance problem we'll revisit the design.
\end{itemize}



%%-----------------------------------------------------------------%%
\section{Misc text looking for a home}

\subsection*{Linking}

\ToDo{All this linking stuff is now bogus.}

There's an abstract @LinkState@, which is threaded through the linkery
bits.  CM can call @addpkgs@ to notify the linker of packages
required, and it can call @addmods@ to announce modules which need to
be linked.  Finally, CM calls @endlink@, after which an executable
image should be ready.  The linker may link incrementally, during each
call of @addpkgs@ and @addmods@, or it can just store up names and do
all the linking when @endlink@ is called.

In order that incremental linking is possible, CM should specify
packages and module groups in dependency order, ie, from the bottom up.

\subsection*{In-memory linking of bytecode}
When being HEP-like, @compile@ will translate sources to bytecodes
in memory, with all the bytecode for a module as a contiguous lump
outside the heap.  It needs to communicate the addresses of these
lumps to the linker.  The linker also needs to know whether a 
given module is available as in-memory bytecode, or whether it
needs to load machine code from a file.

I guess @LinkState@ needs to map module names to base addresses
of their loaded images, + the nature of the image, + whether or not
the image has been linked.

\subsection*{On disk linking of object code, to give an executable}
The @LinkState@ in this case is just a list of module and package
names, which @addpkgs@ and @addmods@ add to.  The final @endlink@
call can invoke the system linker.

\subsection{Finding out about packages, dependencies, and auxiliary 
            objects}

Ask the @packages.conf@ file that lives with the driver at the mo.

\ToDo{policy about upward closure?}



\ToDo{record story about how in memory linking is done.}

\ToDo{linker start/stop/initialisation/persistence.  Need to
      say more about @LinkState@.}


\end{document}


